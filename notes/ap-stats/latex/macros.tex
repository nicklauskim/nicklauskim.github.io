\documentclass{article}
\usepackage[margin=1in]{geometry}
\usepackage{amsmath}
\usepackage{amssymb}
\usepackage{amsthm}
\usepackage{graphicx}
\usepackage{booktabs}
\usepackage{hyperref}
\usepackage{float}
\usepackage{caption}
\usepackage{subcaption}
\usepackage{tikz}
\usetikzlibrary{positioning}
%\usepackage{newtxtext,newtxmath} % Times-like font
\usepackage{mathpazo} % Palatino font
%\usepackage{libertinus} % Libertinus font
\usepackage{pgfplots}
\pgfplotsset{compat=1.17} % or 1.18, etc.
\usepackage{pgf-pie} % pie charts
\usepgfplotslibrary{statistics}
\usepackage{enumitem}
\usepackage{tcolorbox}
\tcbuselibrary{breakable}
\usepackage{fancyhdr}
\pagestyle{fancy}
\fancyhf{}

\theoremstyle{definition}
\newtheorem{definition}{Definition}[section]
\newtheorem{example}{Example}[section]
\newtheorem{properties}{Properties}[section]
\theoremstyle{plain}
\newtheorem{theorem}{Theorem}[section]

\newenvironment{flushlefttab}{
  \begin{flushleft}
  \begin{tabular}{@{} l p{0.75\textwidth} @{}}}
  {\end{tabular}
  \end{flushleft}}



\setlength{\parindent}{0pt}


